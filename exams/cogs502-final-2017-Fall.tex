\documentclass[11pt]{exam}

\usepackage{umut}
\usepackage{mathptmx}
\usepackage{uprog}

\printanswers

\pagestyle{headandfoot}
	\lhead{Cogs 502 -- Prog. \& Log. \\ Fall 2017}
	\chead{Final Exam}
\rhead{Jan 16, 2018}
\lfoot{}
\pointname{\%}


\begin{document}
\qformat{\bf Question \thequestion%
\ifthenelse{\equal{\thepoints}{}}{}{\quad (\thepoints)} \hfill}

\makebox[\textwidth]{Name of the Student:\enspace\hrulefill}
\vspace{10pt}
\begin{center}
\fbox{\parbox{6in}{\bf\centering 4 questions in 150 minutes}}
\end{center}
\vspace{10pt}

\begin{questions}

\question[20]
Given the following information:

\begin{quote}
If wisdom and patience are useful, then there is hope.\\
If there is hope, there is trouble.\\
Patience is useful and there is no trouble.
\end{quote}

What about wisdom, can we decide whether it is useful or not? Motivate your
answer by constructing a truth-tree.
\newpage
\makebox[\textwidth]{Name of the Student:\enspace\hrulefill}
\vspace{10pt}
\question
{\bf Definition:}\\
Given a set $A$ and a relation $R\subseteq A\times A$.
We say that $R$ is \emph{connected} if and only if for every \emph{distinct} $x$
and $y$ in $A$, $(x,y)$ or $(y,x)$ or both are in $R$.

\begin{parts}
\part[5]
Given $A = \{1,2,3,4\}$, what is the smallest number of elements a connected
relation $R$ on $A$ can have. Give such an example relation.
\vspace*{50pt}
\part[5]
Given $A = \{1\}$, is the relation $R=\emptyset$ connected or not?
\vspace*{30pt}
\part[10]
Below is a recursive function that takes two arguments. The first argument, \pyv{baseset}, is a list of integers without repetitions; therefore it represents a set. The second argument, \pyv{relation}, is a binary relation defined over \pyv{baseset}. Binary relations are represented as lists of two element lists. An example call of the function below would be \pyv{connected([1,2,3,4],[[1,3],[3,1]])}.

The code needs some corrections to be able to work properly. What are they?

\begin{ucodeframe}
\inputpygments{python}{code/connected-wrong.py}
\end{ucodeframe}

\end{parts}
\newpage
\makebox[\textwidth]{Name of the Student:\enspace\hrulefill}
\vspace{10pt}
\question[30]
Write a two argument function \pyv{rotate} that takes a list \pyv{s} and an
integer \pyv{n} and rotates the list \pyv{s} \pyv{n} places. By rotation what we mean
	is this:

\pyv{[1,2,3,4]} rotated 1 place is \pyv{[2,3,4,1]}

\pyv{[1,2,3,4]} rotated 2 places is \pyv{[3,4,1,2]}

\pyv{[1,2,3,4]} rotated 5 place is \pyv{[2,3,4,1]}

and so on.
\fillwithlines{\stretch{1}}
\newpage
\makebox[\textwidth]{Name of the Student:\enspace\hrulefill}
\vspace{10pt}
\question[30]
Write a function that takes a possibly nested list and returns a non-nested list.
It gives \Verb+[1,2,3,4,5]+ for \Verb+[[1,2],3,[4],5]+, or \Verb+[1,2,3]+ for
\Verb+[[1],[[2]], [[[3]]]]+, and so on.   

Assume you have a function \pyv{is_list()}, which takes a single argument and returns \pyv{True} if the argument is a list, and \pyv{False} otherwise. Take this function as given; you do NOT have to write it.

\fillwithlines{\stretch{1}}
\end{questions}
\end{document}
