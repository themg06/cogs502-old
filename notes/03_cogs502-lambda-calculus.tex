\documentclass[11pt]{article}
\usepackage{natbib,unatbib}
\usepackage[nohide,twocolumn]{ulecnot}

\usepackage{bussproofs}


\pagestyle{fancy}
\lhead{COGS 502 -- Programming and Logic}
\chead{$\lambda$-Calculus}
\rhead{Updated \it \today}
\lfoot{Umut \"Ozge}
\cfoot{}
\rfoot{Page \thepage/\pageref{LastPage}}
\setlength{\headheight}{13.6pt}

\usepackage{tikz-qtree}

\begin{document}


%\section{Lambda operator, bondage, freedom, substitution}
\section{Expressions}

\ezimeti{

\item Our basic notions here are \uterm{expression} and \uterm{evaluation}.

\begin{udefinition}[Expression]
Let $A=\{ a,b,\ldots,z\}$ be the set of names.
\begin{itemize}
\item[i.] all names are expressions; 
\item[ii.] if $\omega_1$ and $\omega_2$ are expressions, so is $(\omega_1\omega_2)$;
\item[iii.] if $\omega$ is an expression and $\alpha$ is a name, then
$(\lambda\alpha.\omega)$ is an expression.
\item[iv.] nothing else is an expression.
\end{itemize}
\end{udefinition} 

\item Some example expressions:
\renewcommand{\arraycolsep}{6pt}
\renewcommand{\arraystretch}{2}

$$
\begin{array}{cccc}
x & (xy) & (x(yz))  & ((xy)z) \\
(\lambda x.x) & (\lambda y.(\lambda x.x)) & (\lambda z.(x(\lambda y.(yz)))) &
(x(\lambda z.(\lambda y.(yz))))\\
x(\lambda x.x) & (\lambda y.(\lambda x.x))(\lambda x.x) & (\lambda y.(\lambda x.x))(\lambda x.x)(xy)& (x(yz))((xy)z)
\end{array}
$$

\item Note that in lambda calculus we write $f x$ rather than the usual $f(x)$
to represent the application of function $f$ to the argument $x$.


\item[] {\bf Notational conventions:}

\ezimeti{
\item[i.] Omit the outermost parentheses:
	
$$
\begin{array}{cccc}
x & xy & x(yz)  & (xy)z \\
\lambda x.x & \lambda y.(\lambda x.x) & \lambda z.(x(\lambda y.(yz))) & x(\lambda z.(\lambda y.(yz)))
\end{array}
$$

\item[ii.] Parentheses associate to left:    

\begin{align*}
fxyz & \equiv (((fx)y)z)
\end{align*}

\item[iii.] Stacked lambdas:
\begin{align*}
(\lambda f.(\lambda x.(f(fx))))  &\equiv \lambda f \lambda x.(f(fx)) \\
& \equiv \lambda f x.(f(fx))
\end{align*}
}

\item The notions of bondage, freedom and substitution we covered in predicate
logic apply here as well.

\section{$\beta$-reduction}

\item An expression of the form $(\lambda \alpha.\gamma)\omega$ is called a
\uterm{$\beta$-redex}. 


\item It can be \uterm{$\beta$-reduced} to $\subs{\gamma}{\omega}{\alpha}$,
called a \uterm{reduct}, if $\omega$ is free for $\alpha$ in $\gamma$.

\item  A lambda expression can be reduced by turning all the redexes to reducts,
which results in a \uterm{$\beta$-normal} form.

\item Some example reductions:
\begin{align*}
(\lambda f.fx)g  & \breduce gx \\
(\lambda f.fx)ga  & \breduce gxa \\
(\lambda f.fx)(ga) & \breduce gax \\
(\lambda f\lambda x.fx)g a & \breduce ga \\
%(\lambda x\lambda y \lambda z.x(yz))f & \breduce \\
\end{align*}

\item There may be more than one redex in a
expression:

\begin{align*}
(\lambda x.y)((\lambda z.zz)(\lambda w.w)) & \breduce (\lambda x.y)((\lambda w.w)(\lambda w.w))\\
											& \breduce (\lambda x.y)(\lambda w.w)
											& \breduce y 
\end{align*}

\item Another reduction of the same expression would be:

\begin{align*}
(\lambda x.y)((\lambda z.zz)(\lambda w.w)) & \breduce y 
\end{align*}

\item The first is called the \uterm{applicative order} reduction; the second is
called the \uterm{normal order} reduction.

\item Reduce the following expressions:
\begin{align*}
(\lambda x. m x)j\\
(\lambda y. y j)m\\
(\lambda x.\lambda y. y(y x))jm\\
(\lambda y.y j)(\lambda x. m x)\\
(\lambda x. xx)(\lambda y. yyy)
\end{align*}







% \item Reduce the following expression in both applicative and normal order:
% 
% \begin{align*}
% (\lambda p_1\lambda p_2. p_1 0 \land p_2)(\lambda x. x\not= 0)((\lambda y.\frac{5}{y} = 0)0)
% \end{align*}


}
\section{Lambda calculus in action: some examples}

\subsection{Logic}
\ezimeti{
\item Let's define the truth values:

\begin{align*}
\combf{T} \equiv \lambda x\lambda y.x\\
\combf{F} \equiv \lambda x\lambda y.y
\end{align*}

\item Verify that $\lambda x\lambda y.yxy$ behaves like \emph{and} in prefix
notation (i.e. $\land p q$). 

\item Can you think of lambda expressions for $\lor$ and $-$?
\item What about an if-then-else function that applies to the test, a function
to execute if the test is true and a function to execute if the test is false.     

}
\subsection{Arithmetic}

\ezimeti{

\item Number can be represented as lambda expressions in the following way:
\begin{align*}
0 & \equiv \lambda f\lambda x.x\\
1 & \equiv \lambda f\lambda x.fx\\
2 & \equiv \lambda f\lambda x.f(fx)\\
3 & \equiv \lambda f\lambda x.f(f(fx))\\
\vdots
\end{align*}


\item A successor function which returns $n+1$ given $n$ is:
\begin{align*}
\mathbf{S}\equiv \lambda a\lambda f\lambda x. f(afx)
\end{align*}

\item Here are addition and multiplication (again in prefix notation); verify
that they do what they are meant to do. 

\begin{align*}
\mathbf{+} & \equiv \lambda a\lambda b\lambda f\lambda x.af(bfx)\\
\mathbf{\times} & \equiv \lambda a\lambda b\lambda f.a(bf)
\end{align*}


}

\end{document}
