\documentclass[11pt]{article}

\usepackage[nohide,twocolumn]{ulecnot}
\usepackage{natbib,unatbib}
\pagestyle{fancy}
\lhead{COGS 502 -- Programming and Logic}
\chead{Mathematical Preliminaries}
\rhead{Last update \it \today}
\lfoot{Umut \"Ozge -- \href{mailto:umozge@metu.edu.tr}{\nolinkurl{umozge@metu.edu.tr}}}
\cfoot{}
\rfoot{Page \thepage/\pageref{LastPage}}
	\setlength{\headheight}{13.6pt}

\begin{document}
\tableofcontents

\section{Sets}

\subsection{Basic notions}
\begin{itemize}

\item A fundamental activity in science is to characterize classes\footnote{We
use ``class'' in a non-technical, non-set-theoretic sense.} of objects.
Physics: body, electrical charge, spin; 
Biology: living cell, protein folding; 
Sociology: identity, social class; 
all these are classes of concrete or abstract objects.
Any examples from Cognitive Science?

\item The mathematical construct \uterm{set} provides the clearest model of this
activity.

\item Listing the objects belonging to it is the most basic and intuitive way of
specifying a set. You simply list the objects within curly braces, separated by
commas. Here is an example: 

\[
\{\text{``Mu\c s''},\text{``Van''}\}
\]

\item You can pick a symbol to name your set, say $A$:

\[
A=\{\text{``Mu\c s''},\text{``Van''}\}
\]

\item When an object $x$ belongs to a set $X$, we say that $x$ is an
\uterm{element} or \uterm{member} of $X$, symbolically,\footnote{Note that the
case of letters is important in mathematics and programming.} 

\[
x\in X 
\] 

\item[] so
\[
\text{``Van''} \in A 
\]

\item Non-membership is stated via `$\notin$':
\[
\text{``Seattle''} \notin A
\]

\item For any object $x$ and any set $Y$, exactly one of the following holds: $x
\in Y$, $x\notin Y$. There are no cases in between.

\item An object may belong to more than one set. For instance the word ``Mu\c
s'' is an element of many other sets in addition to $A$ above, say the following
set $B$:

\[
B=\{2,\text{``Mu\c s''},\text{``Artvin''}, \text{``France''}\}
\]

\item Our example set $B$ illustrates that there is no
requirement that the elements of a set should be ``similar'' in an intuitive
sense. $B$ has different kinds of elements, and this is perfectly OK. However in
almost all mathematical uses of sets there will be a unifying property. 

\item We said the same object can belong to more than one set, however, an
object may belong to a particular set only once. Therefore 
$\{\text{``Mu\c s''},\text{``Van''},\text{``Mu\c s''}\}$ is no different than
$\{\text{``Mu\c s''},\text{``Van''}\}$. In other words, repetitions do not count
in a set.

\item The order of elements does not matter in sets. Absolutely no difference between
$\{\text{``Mu\c s''},\text{``Van''}\}$ and $\{\text{``Van''},\text{``Mu\c
s''}\}$.

\item The set with no elements is \uterm{the empty set}, denoted as
`$\emptyset$'. We say \emph{the} empty set, because there is one and only one
empty set.\footnote{Why this is so will get clarified in Section~\ref{eqsec},
where we discuss equality.}

\item Sets can have sets as elements. Therefore sets themselves are objects.
The following is a legitimate set:


\[
C=\{2,\{3,4\},\text{``Mu\c s''},\{\text{``Artvin''}, \text{``France''}\}\}
\]

When you need to be more specific in your use of ``object'', you can call
objects that are not sets, \uterm{atomic} objects, and the rest,
\uterm{non-atomic} objects.

\hrulefill 
\begin{uexercise}\label{ex-part} Imagine you are running an
experiment, in which  you will obtain some measurements from trials. You run a
trial and you obtain an integer, you run another one, you have another integer.
You don't know in advance the range of possible values for
measurements, they can be any integer. Imagine further that you need to
accumulate measurements somewhere, so that you can sum them up, or do other
things, after you finish experimenting.  You also want to keep which measurement
comes from which trial. For this task, you are allowed to use sets and integers,
and nothing else.  How would you do this?  

\hyperref[ex-part-sol]{\qed}
\end{uexercise} 
\hrulefill

\item So far, so good with representing sets by listing their elements.
This is, fortunately, not the only way of representing sets. You can
characterize a set by giving a property that uniquely identifies the elements of
that set. For instance, the set $A$ above could have been characterized as the
set of Turkish city names with less than
four letters. This would be equivalent to listing the elements that fulfil the
given criterion.

\item The listing method is sometimes called ``definition by
extension'', and the common property method is called ``definition by
intension'' \cttxp[12]{russell19}.\footnote{``Intension'' and ``extension'' are
important concepts that will come up again in the future. But for now we need
them simply as names to call two different ways of characterizing sets. Do not
try to memorize them.}

\item Definition by intension (aka\footnote{`Also known as.'} intensional
definition) is ``superior'' to extensional definition. The superiority comes
from the fact that all the sets defined by extension can be defined by intension
as well.\footnote{Given any set $A$, there is at least one property shared by
all the members of $A$: ``member of $A$''.} However, the converse does not
hold, namely you cannot give an extensional definition for every set you defined
by intension.

\item[] First, there are sets where you may not identify all the elements
belonging to it due to lack of necessary means; take, for instance,  the set of
all the carbon based molecules in the universe, in which case it is logically
possible, but practically impossible, to enumerate the members.

\item[] Second, there are sets where it is simply impossible to identify all the
elements belonging to it, because regardless of how many elements you
list, there will always remain elements left out, actually infinitely
many of them. These are infinite sets.

\item Both problems are addressed in the same way, and it is actually very
seldom that you list the elements of a set. Instead, as we already started to
see above, one provides a \uterm{decision procedure} or a number of
\uterm{membership criteria}, which gives you a ``yes'' or ``no'' answer, for any
given object, according to whether it is the element of the set or not.

\item[]Here are two very common \uterm{infinite} sets, the set of natural numbers and the set of integers:


\[
\mathbb{N} = \{0,1,2,3,\ldots\}
\]

\[
\mathbb{Z} = \{\ldots,-3,-2,-1,0,1,2,3,\ldots\}
\]

\item In some cases membership criteria is left implicit. 
Here is a finite set of that sort:
\[
K = \crbr{2,4,6,\ldots,90}
\]

\item[]and here is one which is infinite:

\[
K = \crbr{2,4,6,\ldots}
\]

\end{itemize}


\begin{itemize}
\item Another common method is \uterm{predicate} (or \uterm{set-builder}) notation, which has
some variants:
\end{itemize}


\begin{align}
\setcomp{x}{x \in \mathbb{N} \text{ and }7 < x < 11}
\end{align}
\begin{align}
\setcomp{x \in \mathbb{N}}{7 < x < 11}
\end{align}
\begin{align}
\setcomp{x + y}{x, y \in \mathbb{N}\text{ and }7 < x < 11\text{ and }1 < y < 4}
\end{align}


\begin{itemize}

\item You can also think of a set as a \uterm{rule}. A rule that dictates what
belongs to it and what does not.

\item[] Here is a more thorough specification of membership in the set of natural
numbers -- don't worry if you don't
understand this fully now:


\hrulefill
\begin{udefinition}{What is a natural number?}
\begin{itemize}
\item[i.] 0 is a natural number (written $0\in\mathbb{N}$);
\item[ii.] if $n\in\mathbb{N}$, then  $n+1\in\mathbb{N}$; 
\item[iii.] If something is not a natural number according to (i) or (ii), then
it is not a natural number.
% \footnote{As an exercise in logic, try to tell the
% same thing with \emph{unless}, \emph{only if}, \emph{except}, and \emph{nothing
% else}.  What other ways are there to tell it?  Do these work also in Turkish?}
\end{itemize}
\end{udefinition}
\hrulefill


\hrulefill
\begin{uexercise} \label{ex-pred}
Express in predicate notation,
\begin{enumerate}
\item the set of all numbers that can be obtained by multiplying an even
number with 4.5 and adding to it an odd number multiplied by 2.8. 
\item the set of 8 digit numbers that can be read as legitimate dates in
\mbox{\texttt{DDMMYYYY}} format. An illegitimate date is 99032017.
\item the set of expressions that can be read as a time specification in 24
hour format; e.g.\ 10:30, 21:40, but not 34:01, 12:72 etc.
\end{enumerate}

\hyperref[ex-pred-sol]{\qed}
\end{uexercise}

\begin{uexercise} \label{ex-pred2}
Given $A = \crbr{1,2,3,4}$, list the members of the sets:
\begin{enumerate}
\item\label{ex-pred2-a} $\setcomp{x+y}{x,y\in A \text{ and } x-y \geq 2}$
\item\label{ex-pred2-b} $\setcomp{x+y}{x,y\in A \text{ and } x.y \geq 6}$\footnote{`.' means
multiplication.}
\end{enumerate}
\hyperlink{ex-pred2-sol}{\qed}
\end{uexercise}
\hrulefill
\end{itemize}

\ezimeti{
\item For any set $A$, the \uterm{cardinality} of $A$, depicted as $|A|$, stands
for the number of elements in $A$.

\hrulefill
\begin{uexercise}\label{ex-card}
State the cardinality of the following sets:
\begin{itemize}
\item[a.]$\{1,\{2,\{3,\{4\}\}\}\}$
\item[b.]$\emptyset$
\item[c.]$\{\emptyset\}$
\item[d.]$\{\emptyset,1\}$
\item[e.]$\{\emptyset,0,1\}$
\end{itemize}

\hyperlink{ex-card-sol}{\qed}
\end{uexercise}
\hrulefill
}

\begin{comment}
\ezimeti{ 
\item Before moving on, a word on ``objects''. In this course, we will
deal with (sets of) mathematical objects. 

\item[] A mathematical object is an object that exists in the abstract world of
mathematics. Among these are symbols, numbers, functions, relations, sets, and
so on. Therefore our set of Turkish city names above is ``degenerate'' as its
elements -- the two city names -- are not mathematical objects. But don't worry,
we will see that we can easily give mathematical definitions that can take the
place of city names.
}
\end{comment}

\subsection{Equality, subset, set operations}
\label{eqsec}
\ezimeti{
\item[] {\bf Equality:}

\item[] Two sets are equal if and only if they have the same elements.

\item[] {\bf Subset:}

\item[] For any sets $A$ and $B$, \\ $A$ is a \uterm{subset} of (or included in) $B$, written $A \subseteq B$, if and only if each element in $A$ is also in $B$.

\item[] For any sets $A$ and $B$, \\ $A$ is a \uterm{proper subset} of $B$, written $A \subset B$, if and only if each element in $A$ is also in $B$ and
there is at least one $a \in B$, such that $a \notin A$.

\item Given any set $A$, is $A\subseteq A$?\footnote{Of course; how else could
it be?}

\item Given any set $A$, what can you say about $\emptyset \subseteq
A$?\footnote{We can be confident that the statement is true. If it were not,
then there would have been an element belonging in the empty set without
belonging in $A$, which is impossible, given the empty set is empty. In
mathematics, if a statement cannot be true, it must be false, and vice versa; no
third way.}

\item Given the notion of equality and subsethood, can you see why the following
holds?

\item[] For any sets $A$ and $B$, \\ $A$ is \uterm{equal} to $B$, written $A
= B$, if and only if $A \subseteq B$ and $B \subseteq A$.

\hrulefill
\begin{uexercise}\label{truefalse1}

State whether true or false:\footnote{(\ref{e1}--\ref{e6}) are
from  \cite[p.\ 8]{lewispapadimitriou98}.}

\begin{enumerate}
\item\label{e1} $\emptyset \in \{\emptyset\}$

\item\label{e2} $\emptyset \subseteq \{\emptyset\}$

\item\label{e3}  $\emptyset \subseteq \emptyset$

\item\label{e4} $\emptyset \in \emptyset$

\item\label{e5} $\crbr{2,4} \subseteq \crbr{2,4,\crbr{2,4}}$

\item\label{e6} $\crbr{2,4} \in \crbr{2,4,\crbr{2,4}}$

\item\label{f1}
$\crbr{a} \subseteq \crbr{\crbr{a}}$
\item \label{f2}
$\emptyset \notin \crbr{a,b,c}$
\item \label{f3}
$\crbr{\emptyset} \subseteq \crbr{a,b,c}$
\item\label{f5}
Given any set $A$, $\emptyset \subset A$.
\end{enumerate}
 
\hyperlink{truefalse1-sol}{\qed}
\end{uexercise}

\begin{uexercise}\label{ex-aina}

Give a set $A$ such that there exists an $a$ where both $a \in A$ and
$a\subseteq A$.

\hyperlink{ex-aina-sol}{\qed}
\end{uexercise}

\hrulefill
}



\ezimeti{
\item[] {\bf Set operations:}

\item[]{\bf Union}:

\item[] Given two sets $A$ and $B$, \\ The union of $A$ and $B$, written $A \cup
 B$, is the set of objects that are elements of at least one of $A$ and $B$. 


\item[]{\bf Intersection}:

\item[] Given two sets $A$ and $B$, \\ The intersection of $A$ and $B$, written $A \cap
 B$, is the set of objects that are elements of both $A$ and $B$. 

\item[]{\bf Difference}:

\item[] Given two sets $A$ and $B$, \\ The difference of $A$ and $B$, written $A
- B$, is the set of objects that are elements of $A$ but not $B$. 

\hrulefill
\begin{uexercise} \label{ex-prednot}
Define $A\cup B$,\  $A\cap B$ and $A - B$ in predicate notation.

\hyperlink{ex-prednot-sol}{\qed}
\end{uexercise}
\hrulefill


\item Two sets are \uterm{disjoint} if they have no element in common, their
intersection is $\emptyset$.

\item Here are some properties of set operations:
\item[] 
\renewcommand{\arraystretch}{1.3}
\begin{tabular}{ll}
{\bf Idempotency:} 		& $A\cup A = A$ \\
						& $A\cap A = A$ \\
{\bf Commutativity:}	& $A\cup B  = B\cup A$\\
						& $A\cap B  = B\cap A$\\
{\bf Associativity:} 	& $A\cap (B \cap C) = (A\cap B) \cap C$\\
						 & $A\cup (B \cup C) = (A\cup B) \cup C$\\

{\bf Distributivity:} 	& $A\cap(B\cup C) = (A\cap B)\cup (A\cap C)$\\
						& $A\cup(B\cap C) = (A\cup B)\cap (A\cup C)$\\
{\bf DeMorgan's Laws:} 	& $A - (B\cup C) = (A-B) \cap (A-C)$  \\
					 	& $A - (B\cap C) = (A-B) \cup (A-C)$ 
\end{tabular}

% \begin{comment}
% {\bf Associativity:} 	& $A\kappa (B \kappa C) = (A\kappa B) \kappa C$, for
% $\kappa \in \crbr{\cap,\cup}$\\
% \end{comment}

\item No need to memorize these, they all follow from the basic definitions we
had above. Do not try to memorize anything -- except perhaps some names -- in
this course, it is not the right way to
learn mathematics.  

\item It is possible to take the union or intersection of more than two sets. Given a set of sets $A$:
\begin{align*}
	\bigcup A = \crbr{a\, |\, a \in B\text{ for some }B \in A}\\
	\bigcap A = \crbr{a\, |\, a \in B\text{ for each }B \in A}
\end{align*}

\hrulefill

\begin{uexercise}\label{ex-bigOp}
State whether true or false:
\begin{enumerate}

\item\label{ex-bigOp-x}
$A - (A \cap B) = A - B$
\item\label{ex-bigOp-y}
$\crbr{a,b,\crbr{a,b}} - \crbr{a,b} = \crbr{a,b}$

\item\label{ex-bigOp-a} If $\bigcap X = \emptyset$, then there exists $x,y\in X$ such that $x\cap
y = \emptyset$.
\item\label{ex-bigOp-b} If there exists $x,y\in X$ such that $x\cap
y = \emptyset$, then $\bigcap X = \emptyset$.
\end{enumerate}

\hyperlink{ex-bigOp-sol}{\qed}
\end{uexercise}

\begin{uexercise}\label{ex-tsubs}
For any set $A$, what is the union of all the subsets of $A$  two
elements?

\hyperlink{ex-tsubs-sol}{\qed}
\end{uexercise}

\begin{uexercise}[*] \label{ex-dist} 
Define distributivity by using variables over set operations.

\hyperlink{ex-dist-sol}{\qed}
\end{uexercise}

\begin{uexercise}[*] \label{ex-demorg} 
Show that DeMorgan's Laws hold.  Concentrate on what is required for
two sets to be equal.

\hyperlink{ex-demorg-sol}{\qed}
\end{uexercise}
\hrulefill
}

\subsection{Power set and partition}

\ezimeti{
\item[] {\bf Power set:}

\item[] For any set $A$, \\ The \uterm{power set} of $A$, written
$\mathcal{P}(A)$, or $\mathrm{Pow}(A)$, is the set of all subsets of $A$.

\item[] {\bf Partition:}

\item[] For any set $A$, \\ A \uterm{partition} of $A$, is a set $\Pi \subseteq
\mathcal{P}(A)$ such that:

\ezimeti{
\item[i.] $\emptyset \notin \Pi$;
\item[ii.] any two distinct members of $\Pi$ are disjoint; 
\item[iii.] $\bigcup \Pi = A$.
}

\hrulefill
\begin{uexercise}\label{ex-partit}
State whether true or false:\footnote{(\ref{e7}--\ref{e10}) are
from  \cite[p.\ 8]{lewispapadimitriou98}, (\ref{e11}) is from
\cite[p.\ 34]{nesinSKK}.}

\begin{enumerate}

\item\label{ex-partit-x} 
$\crbr{\crbr{a,\crbr{b}},\crbr{c,d}}$ is not a partition of $\crbr{a,\crbr{b},c,d}$

\item\label{e7} $\crbr{2,4} \subseteq \mathcal{P}(\crbr{2,4,\crbr{2,4}})$ \hfill 
	
\item\label{e8} $\crbr{2,4} \in \mathcal{P}(\crbr{2,4,\crbr{2,4}})$ 

\item\label{e9} $\crbr{\crbr{2,4}}\in \mathcal{P}(\crbr{2,4,\crbr{2,4}})$ 

\item\label{e10} $\crbr{\crbr{2,4}}\subseteq \mathcal{P}(\crbr{2,4,\crbr{2,4}})$ 

\item\label{e11} If $a,b \in A$, then $\crbr{\crbr{a},\crbr{a,b}}\in
\mathcal{P}(\mathcal{P}(A))$ 

\end{enumerate}

\hyperlink{ex-partit-sol}{\qed}
\end{uexercise} 

\begin{uexercise}\label{ex-empow}
What is the power set of $\crbr{\emptyset,\crbr{\emptyset}}$?

\hyperlink{ex-empow-sol}{\qed}
\end{uexercise}

\begin{uexercise}\label{ex-tpart}
Let $A=\crbr{a,b,c,d}$; give all the partitions of $A$ whose members has
\emph{at most} 2 members.

\hyperlink{ex-tpart-sol}{\qed}
\end{uexercise}

\begin{uexercise}\label{ex-part-pred}
Write the following sets in predicate notation:
\begin{enumerate}
\item\label{ex-part-pred-a}
The power set of a given set.
\item\label{ex-part-pred-b}
The subsets of a given set with less then three elements.
\end{enumerate}

\hyperlink{ex-part-pred-sol}{\qed}
\end{uexercise}

\hrulefill
}


\section{Relations}
\ezimeti{
\item Up to now we saw objects and their collections, sets. Mathematics
extensively deals with \uterm{relations} between objects. 

\item We have already seen some relations; one is the membership relation designated
with `$\in$'. It relates objects with sets that they are members of, say, $a\in
\crbr{a,b,c}$. Membership is a \uterm{two-place} (or \uterm{binary}) relation,
since it relates two things -- the technical term for things related is
\uterm{relata}, and \uterm{relatum} for singular. Generally there can be
\emph{n}-ary relations. Some examples?  

\item[] The mathematical way of representing binary relations is -- rather
weirdly you may find -- to form the set of pairs of related objects. For
instance, `less than' relation is the set of pairs of numbers, where the first
number in each pair is less than the second.

\item[] Therefore, in order to represent relations, we need a way to represent pairs.

\item Now comes a new type of object:
\item[] {\bf Tuple:}
\[
(o_1,o_2,\ldots,o_n)
\]
\item[] Order and repetition matter: $(a,a,b)\, \neq\, (a,b)\, \neq\, (b,a)$.
\item[] Terminology: ``ordered pair'' for 2-tuple, ``ordered triple'' for
3-tuple, and so on.



\item The Cartesian \uterm{product} of two sets:
\[
A \times B = \crbr{(a,b)|\, a \in A,\, b \in B}
\]


generally, the product of $n$ sets:


\[
A_1 \times A_2 \times\ldots\times A_n = \crbr{(a_1,a_2,\ldots,a_n) | a_i \in A_i}
\]


\hrulefill
\begin{uexercise}\label{ex-prod}
\ezimeti{
\item[]
\item[a.]Give the product of $\crbr{1,2,3}$ and $\crbr{a,b,c}$.
\item[b.]Give the product of $\crbr{a,b,c}$ and $\emptyset$.
\item[*c.] Think of a way to represent ordered pairs as sets.
}
\end{uexercise}
\hrulefill


\item A \uterm{binary relation} on sets $A$ and $B$ is a subset of $A \times B$.  

\item[] For instance $\crbr{(a,\crbr{a}),(b,\crbr{a})}$  is a binary relation on
$\crbr{a,b,c}$ and $\crbr{\crbr{a},\crbr{b},c}$.

\item Generally an $n$-ary relation on sets $A_1,\ldots,A_n$ is a subset of
$A_1\times\ldots\times A_n$.


\item There is no requirement that $A_i$s  be distinct, when they are the same,
abbreviate $A_1\times\ldots\times A_n$ as $A^n$.

\item[] For instance `less than' is a subset of $\mathbb{N}^2$, namely:

\[
\crbr{(i,j) |\, i,j \in \mathbb{N} \text{ and } i < j}
\]

\item The \uterm{domain} of a relation $R\, \subseteq\, A\times B$:
\[
	\crbr{a\,|\, \text{there is a } b\in B \text{ such that }  (a,b)\, \in\, R}
\]

\item The \uterm{range} of a relation $R\, \subseteq\, A\times B$:
\[
	\crbr{b\,|\, \text{there is an } a\in A \text{ such that }  (a,b)\, \in\, R}
\]

\item Any binary relation $R\subseteq A\times B$ has an \uterm{inverse} $R^{-1}
\subseteq B\times A$ defined as:
\[
R^{-1}=\crbr{(b,a)\,|\, (a,b) \in R}
\]

}

\subsection{Some properties of binary relations}
\ezimeti{
\item Certain types of relations are of special interest due to way they are
structured.

\item Let's focus on binary relations on a single set, $R\subseteq A\times A$.


\item[]{\bf Reflexivity:}

A relation $R\subseteq A\times A$ is \uterm{reflexive} if and only if for each
$x \in A$, $(x,x) \in R$.

A relation is \uterm{nonreflexive}, if it is not reflexive.

A relation $R\subseteq A\times A$ is \uterm{irreflexive}, if for each $(x,y) \in
R$,  $x\neq y$.


\item[] {\bf Symmetry:}

A relation $R\subseteq A\times A$ is \uterm{symmetric} if and only if for each
$(x,y) \in R$, $(y,x)$ is also in $R$.

A relation $R\subseteq A\times A$ is \uterm{nonsymmetric} if and only if 
for some $(x,y) \in R$,  $(y,x)$ is not in $R$.

A relation $R\subseteq A\times A$ is \uterm{asymmetric} if and only if for each
$(x,y) \in R$, $(y,x)$ is not in $R$.

A relation $R\subseteq A\times A$ is \uterm{anti-symmetric} if and only if
whenever $(x,y)$ and $(y,x)$ are  in $R$,  then $x=y$.

\item[] {\bf Transitivity:}

A relation $R\subseteq A\times A$ is \uterm{transitive} if and only if
whenever $(x,y)$ and $(y,z)$ are  in $R$,  then $(x,z)$ is also in $R$.

A relation is \uterm{nontransitive}, if it is not
transitive.

A relation $R\subseteq A\times A$ is \uterm{intransitive} if and only if
for no pair $(x,y)$ and $(y,z)$  in $R$,  $(x,z)$ is in $R$.


\item[] {\bf Connectedness:}


A relation $R\subseteq A\times A$ is \uterm{connected} if and only if for each
$x$, $y \in A$ where $x\neq y$, either $(x,y)$, $(y,x)$ or both are in $R$.


\item Our final special type of relation is \uterm{equivalence relation}, which
is reflexive, symmetric and transitive.

\hrulefill
\begin{uexercise}\label{ex-relprop}
\ezimeti{
\item[]
\item[a.] State the properties of the following relations defined over set of
humans: `spouse', `ancestor', `sister', `sibling', `father', `child', `admire',
`identical', `older than', `older than or at the same age as', `has the same hight as', `has the same biological
father', `has the same cousin'.
\item[b.] 
Give other examples for each type of relation from $\mathbb{N}^2$ and/or
$H^2$, where $H$ is the set of humans.
}
\end{uexercise}
\hrulefill
}
\section{Functions}

\ezimeti{
\item A \uterm{function} is a special type of relation.

\item A function from set $A$ to set $B$ is a relation $R\, \subseteq\,  A \times B$,\\
such that for each $a\, \in\, A$ there is exactly one pair in $R$ with $a$ as
the first component.

\hrulefill
\begin{uexercise}\label{ex-is}
Is $\crbr{(a,\crbr{a}),(b,\crbr{a}),(c,c)}$  a function 
from $\crbr{a,b,c}$ to $\crbr{\crbr{a},\crbr{b},c}$?
\end{uexercise}
\hrulefill

\item  Letters $f$, $g$, $h$ are usually reserved for representing functions.

\item You can think of a function as a mapping from a set to another, written\\
$f:A\rightarrow B$.

\item[] For an $a\, \in\, A$, $f(a)\, \in\, B$ is called the \uterm{image} of
$a$ under $f$, or simply $f$ of $a$.

\item[] Given any set \sysm{A'\subseteq A}, the image of \sysm{A'} under
\sysm{f}:
\[
\crbr{b\,|\, f(a)=b\text{ for some } a\in A'}
\]

\item[] For any function $f:A\rightarrow B$,\\
the \uterm{domain} of \sysm{f}, denoted by Dom(\sysm{f}), is\ldots\\
the \uterm{range} of \sysm{f} is denoted by Ran(\sysm{f}), and
\sysm{\mathrm{Ran}(f)\ldots}


\item[] Seen as a mapping, the condition for functionhood is that the function
maps each and every element in its domain to one and only one (= exactly one)
element in its range.

\hrulefill
\begin{uexercise}\label{ex-which}
Which of these are functions (where $y$ would be the image of $x$): `$x$ is the mother of $y$', `$x$ is a child of
$y$', `$x$ is $y$ years old', `$x$ is the age of $y$', 
`$x$ is the capital of $y$', `the capital of $x$ is $y$', `$x$ is the same person as $y$'?
\end{uexercise}
\hrulefill

 \item When the domain of a function consists of tuples we omit the parentheses
 around tuples: 
 
\item[] For $f: A_1\times,\ldots,A_n \rightarrow B$,
  instead of $f((a_1,\ldots,a_n))$ for $a_i\in A_i$, we write $f(x_1,\ldots,x_n)$.
 
\item[] The objects \sysm{a_1,\ldots,a_n} are called the \uterm{arguments} of
$f$. The object $b\in B$ that $f$ maps these arguments to is the \uterm{value} of
$f(a_1,\ldots,a_n)$.


\hrulefill
\begin{uexercise}\label{ex-inverse}
Unlike relations, the inverse of a function may not be a function. Why?
\end{uexercise}
\hrulefill


\item Given two relations $Q$ and $R$,
%where Ran($R$) = Dom($Q$), 
the \uterm{composition} of them, $Q\circ R$ is the relation,
\[
\crbr{(a,c)\,|\, (a,b) \in R \text{ and } (b,c) \in Q \text{ for some } b}
\]


\item The composition of two functions $f:A\rightarrow B$ and $g:B\rightarrow C$,
denoted by $g\circ f$, is a function $h:A \rightarrow C$, such that
\[
h(a) = g(f(a)) \text{ for each } a \in A
\]


\item Some special types of functions:

\item[] {\bf Constant} functions:\\
any function $f:A\rightarrow B$ such that for all $a\, \in A$, $f(a) = c$ for
some $c\in B$.

\item[] A function $f:A\rightarrow B$ is \uterm{onto} \sysm{B} (or simply onto)
if \sysm{\mathrm{Ran}(f) = B}.

\item[] A function $f:A\rightarrow B$ is \uterm{one-to-one} if for any
\sysm{a_1,a_2 \in A}, \sysm{f(a_1) \neq f(a_2)}.\footnote{What's wrong with
this?}

\item[]A function $f:A\rightarrow B$ is a \uterm{bijection} (or
\uterm{one-to-one correspondence}), if it is one-to-one and onto.


\item Let's think of some examples for each type.

}

\begin{uexercise}\label{ex-equi}
Every equivalence relation defines a partition on the set it is
defined over, where each cell of the partition is called an \uterm{equivalence
class}. Can you see how/why?
\end{uexercise}


% \section*{Optional Material}
% \subsection*{Russell's Paradox}
% 
% Russell's paradox is a discovery made by Bertrand Russell which is of historical
% importance for set theory and mathematics in general. The paradox is related to
% the Axiom of Specification, which says that:
% \begin{align}
% \text{Given any set } A \text{ and a specification } S \text{, you can construct
% another set }\\  B = \setcomp{x\in A}{ S \text{ holds for } x} 
% \end{align}

\newpage
\appendix
\section{Answers for selected exercises}

\begin{enumerate} 

\item[\ref{ex-part}]\label{ex-part-sol} Number your trials from 0 to $k$; put $n$
number of braces around the score of the $n$th trial and put everything in a
set. Another method is to bring together a trial number and its measurement into
a set, but putting one of them, say the measurement, in another set. E.g.\ trial
10 with measurement 20 is represented as $\crbr{10,\crbr{20}}$. Putting all such
sets in another set you can collect them without losing any information.

\item[\ref{ex-pred2}]\hypertarget{ex-pred2-sol}{} 
\ref{ex-pred2-a}: $\crbr{4,5,6}$ 
\ref{ex-pred2-b}: $\crbr{5,6,7,8}$ 

\item[\ref{ex-card}]\hypertarget{ex-card-sol}{} 2, 0, 1, 2, 3.

\item[\ref{truefalse1}]\hypertarget{truefalse1-sol}{}

\ref{e1}: T, 
\ref{e2}: T, 
\ref{e3}: T, 
\ref{e4}: F, 
\ref{e5}: T, 
\ref{e6}: T, 
\ref{f1}: F, 
\ref{f2}: T, 
\ref{f3}: F, 
\ref{f5}: F (when $A=\emptyset$)

\item[\ref{ex-aina}]\hypertarget{ex-aina-sol}{}

$A=\{ \emptyset\}$, $a=\emptyset$, or anything like $A=\{x,\{ x\}\}$, $a=\{ x\}$.

\item[\ref{ex-prednot}]\hypertarget{ex-prednot-sol} 
$A\cup B = \setcomp{x}{x\in A \text{ or } x \in B}$ \\
$A\cap B = \setcomp{x}{x\in A \text{ and } x \in B} $ \\
$A - B = \setcomp{x}{x\in A \text{ and } x \not\in B}$


\item[\ref{ex-bigOp}]\hypertarget{ex-bigOp-sol}{}

\ref{ex-bigOp-x}. True 

\ref{ex-bigOp-y}. False 

\ref{ex-bigOp-a}. False; every pair of sets in $X$ may have something in common,
without there being any element that is common to all the sets in $X$. For
instance, take $X=\crbr{\crbr{a,b},\crbr{b,c},\crbr{a,c}}$.

\ref{ex-bigOp-b}.
True. Assume that it is false; which means there are $x,y\in X$ and
$x\cap y = \emptyset$, and $\bigcap X \neq \emptyset$; then there is a $z \in
\bigcap X$; then this $z$  must be both in $x$ and $y$, and therefore it must be
in $x\cap y$. But we started by saying that $x\cap y = \emptyset$. Therefore
there is no way that the statement be false, it must be true.

\item[\ref{ex-tsubs}]\hypertarget{ex-tsubs-sol}{}

If $A$ has less then 2 elements, the union is $\emptyset$, otherwise the union
is $A$.

\item[\ref{ex-dist}]\hypertarget{ex-dist-sol}{}

$A\,\alpha\, (B\, \beta\, C) = (A\,\alpha\, B) \beta\, (A\,\alpha\, C)$, for
$\alpha, \beta \in \crbr{\cap,\cup}$ and $\alpha\neq\beta$.

\item[\ref{ex-demorg}]\hypertarget{ex-demorg-sol}{}
We discuss the exercise deliberately in a rather
roundabout way to increase our familiarity with sets and logical inference. Take\\
$A - (B\cup C) = (A-B) \cap (A-C)$\\
if there is an $a$ in the left hand side (LHS), it must be something in $A$ but
neither in $B$ nor $C$. The question is is $a$ guaranteed to be in the RHS as
well? For $a$ to be in RHS it must be present in both parts of the intersection.
Suppose it is not, then it must be missing from at least one of the parts of the
intersection. For it to be missing in $A - B$, it must either be not in $A$ or
it must be in $B$. Both possibilities contradict our assumption that $a$ is in
LHS. The second way that $a$ is missing from the intersection on RHS, it must be
missing from $A - C$. Then in this case it is either not in $A$ or it is in $C$.
Again both possibilities clash with our initial assumption. Therefore there is
no way that $a$ is in LHS but not in RHS; it is simply impossible. You can prove
that $a$ must be in LHS, if it is in RHS, in a similar fashion; thereby
finalizing the proof of the first DeMorgan's Law. The proof for the second
would be very similar to the first one. 

\item[\ref{ex-partit}]\hypertarget{ex-partit-sol}{}
\ref{ex-partit-x}: F,
\ref{e7}: F, 
\ref{e8}: T, 
\ref{e9}: T, 
\ref{e10}: T 


\item[\ref{ex-empow}]\hypertarget{ex-empow-sol}{}

$\crbr{\emptyset,
\crbr{\emptyset},
\crbr{\crbr{\emptyset}},
\crbr{\emptyset,\crbr{\emptyset}}}$


\item[\ref{ex-tpart}]\hypertarget{ex-tpart-sol}{}

To better keep track of things, let us distinguish three cases:

Case 1: Partitions with no member with more than one member. There is only one
such partition:

\begin{itemize}

\item[] $\crbr{\crbr{a},\crbr{b},\crbr{c},\crbr{d}}$

\end{itemize}

Case 2: Partitions with only one member with two members. In this case there are
as many partitions as there are distinct ways of picking 2 members from $A$,
without caring for in which order you pick the members. There are 6 distinct
ways for doing this. (It is OK if you cannot find this number without explicitly
counting the possibilities.) The partitions are:

\begin{itemize}
\item[] $\crbr{\crbr{a,b},\crbr{c},\crbr{d}}$
\item[] $\crbr{\crbr{a,c},\crbr{b},\crbr{d}}$
\item[] $\crbr{\crbr{a,d},\crbr{b},\crbr{c}}$
\item[] $\crbr{\crbr{b,c},\crbr{a},\crbr{d}}$
\item[] $\crbr{\crbr{b,d},\crbr{a},\crbr{c}}$
\item[] $\crbr{\crbr{c,d},\crbr{a},\crbr{b}}$
\end{itemize}

Case 3: Partitions with two members with two members. This case is not much
different from Case 2. The number of possibilities is the same. You simply put
together the sets with one members: 

\begin{itemize}
\item[] $\crbr{\crbr{a,b},\crbr{c,d}}$
\item[] $\crbr{\crbr{a,c},\crbr{b,d}}$
\item[] $\crbr{\crbr{a,d},\crbr{b,c}}$
\end{itemize}

\item[\ref{ex-part-pred}]\hypertarget{ex-part-pred-sol}{}

\ref{ex-part-pred-a}. \sysm{\mathrm{Pow}(A) = \crbr{X\, |\, X \subseteq A}}

\ref{ex-part-pred-a}. Given $A$, \sysm{\crbr{X\, |\, X \subseteq A\, \mathrm{ and }\, |X| < 3}}

\item[\ref{ex-prod}a]
$\crbr{(1,a),(1,b),(1,c),(2,a),(2,b),(2,c),(3,a),(3,b),(3,c)}$.
\item[b] $\emptyset$.
\item[c] A first attempt would be having $(a,b)$ equivalent to
$\crbr{a,\crbr{b}}$. There are cases where this solution would fail. Can you see
when? The definitive solution is to have $(a,b)$ equivalent to
$\crbr{a,\crbr{a,b}}$.

\item[\ref{ex-relprop}a] `admire' is the only non-reflexive relation; `identical'
and `has the same hight', `has the same biological father as'  are equivalence
relations; `has the same cousin' is reflexive, symmetric but non-transitive (why
is it not intransitive?); `older than or at the same age as' is reflexive,
transitive and anti-symmetric\ldots

\item[\ref{ex-is}] Yes.

\item[\ref{ex-which}]
 `$x$ is $y$ years old',
`$x$ is the capital of $y$', `the capital  of $x$ is $y$', `$x$ is the same person as $y$'.


\item[\ref{ex-inverse}] Take the function in Exercise~\ref{ex-is} as an example.

\item[*\ref{ex-equi}] Let $R\subseteq A\times A$ be an equivalence relation in
$A$. Let $g: A \mapsto \mathcal{P}(A)$ be the function defined as $g(a) =
\setcomp{x}{(a,x) \in R}$, for $a\in A$. We need to show that the range of $g$,
which is some collection of sets $\mathcal{G}$, is a partition of $A$. {\bf
First}, we need to show that $\emptyset \not\in \mathcal{G}$. Assume that
$\emptyset \in \mathcal{G}$. Then, given the definition of $g$, there needs to
be an $a \in A$ where there is no $b \in A$ such that $(a,b) \in R$. But given
that $R$ is an equivalence relation, $(a,a) \in R$, providing such a $b$,
resulting in a contradiction. Therefore  $\emptyset \not\in \mathcal{G}$. {\bf
Second} we need to show that for $G_1, G_2 \in \mathcal{G}$, if $G_1 \neq G_2$,
then $G_1\cap G_2 = \emptyset$. Take two such non-identical sets $G_1, G_2 \in
\mathcal{G}$. Assume $G_1\cap G_2 \neq \emptyset$. Then there must be an $a \in
A$ which belongs to both $G_1$ and $G_2$. Take any $b \in G_1$, given the
definition of $g$,  that $R$ is an equivalence relation and $a\in G_1$, $(a,b),
(b,a) \in R$; and as $a \in G_2$, $b\in G_2$ as well. Now pick any $c\in G_2$,
by the same reasoning $c$ is also in $G_1$, establishing that $G_1 = G_2$. This
contradicts with what we had about $G_1$ and $G_2$ in the beginning. Therefore,
whatever two non-identical sets we pick out of $\mathcal{G}$, they are
guaranteed to be disjoint. {\bf Finally}, we need to show that $\bigcup
\mathcal{G} = A$. One way this fail to be true is that  there is an $a\in A$ which does
not belong to any set in $\mathcal{G}$. This is obviously impossible, since
given $R$ is an equivalence relation and $a\in A$, $(a,a) \in R$, and therefore
$g(a)$ is some set in $\mathcal{G}$ such that $a \in g(a)$. As it is also
obvious that there can be no $a \in \bigcup\mathcal{G}$ which is not present in
$A$, we have to conclude that $\bigcup
\mathcal{G} = A$. This finishes our proof that an equivalence relation over a
set defines a partition of that set. Each member of this partition is called an
equivalence class induced by the equivalence relation. 

\end{enumerate}


\bibliographystyle{plain}
\bibliography{ozge}
\end{document}
